\documentclass[11pt,a4paper]{article}



\usepackage[german]{babel}
\usepackage{amsmath}
\usepackage[utf8]{inputenc}

\date{24. Oktober 2014}
\author{Thomas Rölli}
\title{CS102 \LaTeX\ Übung}


\begin{document}
\maketitle

\section{Das ist der erste Abschnitt}
Hier könnte auch anderer Text stehen.
\section{Heute zu Besuch}
Ich war hier! glg Julian :)
\section{Tabelle}
Unsere wichtigsten Daten finden Sie in Tabelle \ref{tab1}.
\begin{table}[h]
\centering
\begin{tabular}{c|c|c|c}
& Punkte erhalten & Punkte möglich & \% \\
\hline
Aufgabe 1 & 2 & 4 & 0.5 \\
Aufgabe 2 & 3 & 3 & 1 \\
Aufgabe 3 & 3 & 3 & 1 \\
\end{tabular}
\caption{Diese Tabelle kann auch andere Werte beinhalten.}
\label{tab1}
\end{table}
\section{Formeln}
\subsection{Pythagoras}
Der Satz des Pythagoras errechnet sich wie folgt: $a^2+b^2=c^2$. Daraus können wir die Länge der Hypotenuse wie folgt berechnen: $c=\sqrt{a^2+b^2}$
\subsection{Summen}
Wir können auch die Formel für eine Summe angeben:
\begin{equation}
s=\sum_{i=1}^{n}i=\frac{n*(n+1)}{2}
\end{equation}
\end{document}

